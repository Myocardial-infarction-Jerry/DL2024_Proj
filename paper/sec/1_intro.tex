\section{Introduction}
\label{sec:intro}

Fine-Grained Visual Classification (FGVC) is an advanced computer vision technique designed to distinguish between highly similar subcategories within broader categories. This classification challenge extends beyond the general task of differentiating between categories like 'cat' and 'dog', delving into more intricate distinctions, such as identifying specific dog breeds, bird species, or car models. Despite substantial advancements leading the field, FGVC continues to present significant challenges:

\begin{itemize}
    \item \textbf{Fine-grained differences:} The distinctions between subcategories in FGVC are often subtle and can be challenging to detect, sometimes even for human observers. For instance, differentiating bird species might rely on minute differences in plumage color, beak shape, or body size.
    \item \textbf{Limited training data:} The datasets used for FGVC typically suffer from small size and significant class imbalance, complicating the development of models that perform well on new, unseen examples. This issue is particularly pronounced for rare or uncommon subcategories, which may be represented by only a handful of examples within the available training data.
\end{itemize}

These obstacles necessitate ongoing research and methodological innovations to push the boundaries of what FGVC can achieve in practical and theoretical applications. Serveral approaches can help address them, like GAN, which generates more available images for the expansion of datasets, and CNN backbone 
(ResNet) or ViT model backbone, which can help models focus on any discriminative regions of an image. Besides, it is inspiring of using CAM to generate heat map with hooked results for model interpretability. What's more, VLM (such as clip-vit-base-patch32) is another good way for dealing with the assignment.

In the following sections, we will review the related work in FGVC and discuss how the proposed approach differs from and improves upon existing methods. We will then describe the proposed methodology in detail and present experimental results to validate its effectiveness.
