\section{Methods}
\label{sec:methods}

%\subsection{Training Techniques}
%In this section, we delineate the various training techniques employed to optimize the performance of our model. These methods are fundamental to ensuring robust and reproducible results.

%\subsubsection{Random Seed}
%To ensure the reproducibility of our experiments, a fixed random seed was set across all stages of model training and evaluation. This practice mitigates the variability in results caused by random initializations and stochastic processes inherent in machine learning algorithms.
\subsection{Optimizer}
The choice of optimizer is critical for the convergence and performance of deep learning models. We experimented with several optimizers, including Stochastic Gradient Descent (SGD), Adam, and RMSprop. Our final choice was based on empirical performance and convergence speed.

\subsection{Lp Regularization}
To prevent overfitting, we incorporated Lp regularization techniques, including L2 (Ridge) and L1 (Lasso) regularization. These techniques add a penalty to the loss function, proportional to the magnitude of the coefficients, thus encouraging the model to maintain simpler and more generalizable parameters.

\subsection{Cross-Validation}
We utilized k-fold cross-validation to rigorously evaluate the performance of our model. This method involves partitioning the dataset into k subsets, training the model on k-1 subsets, and validating it on the remaining subset. This process is repeated k times, with each subset serving as the validation set once. Cross-validation provides a comprehensive assessment of the model’s generalizability.



\subsection{Data Augmentation}
Data augmentation is a widely used technique in machine learning and deep learning to increase the diversity and quantity of training data, ultimately improving the generalization capability of models.New training samples are generated by applying various transformations (such as rotation, flipping, scaling, etc.) to the original data, thus increasing the diversity and quantity of the dataset.Then the model can learn more feature representations, leading to improved performance on unseen data.

\subsection{Generating Extra Dataset}
To augment our training data, we tried to utilize the Deep Convolutional Generative Adversarial Network (DCGAN) and the Diffusion Model. These techniques can generates synthetic data that resembles the real dataset, thereby increasing the diversity and quantity of training examples. This approach is particularly beneficial when the available labeled data is limited.

\subsection{Pretrained Model}
For our initial model weights, we employed a pretrained Vision Transformer. Pretrained models provide a significant advantage by leveraging features learned from large-scale datasets, thereby improving the performance and convergence speed on our specific task.

\subsection{Hyperparameter Tuning}
Hyperparameter tuning was performed using a grid search approach. Key hyperparameters tuned included the learning rate, batch size, and regularization coefficients. The optimal hyperparameters were selected based on validation performance.
%\subsection{Backbone Model}
%We compared the performance of two prominent backbone models: ResNet and Vision Transformer (ViT). ResNet, known for its deep residual learning framework, provides robust feature extraction capabilities, while ViT, leveraging self-attention mechanisms, offers superior performance in capturing global dependencies. Our experiments evaluated these models based on accuracy, computational efficiency, and scalability.

\subsection{Ensemble Learning}
Besides the methods of ResNet and the ViT, we also tried the VLM (Visual-Linguistic Model) to improve the performance of the classifier. So we need to use the Method of Ensemble Learning, utilizing multiple methods as base learners and combining results of these methods, for improving overall predictive performance and robustness of the classifier.

%\subsection{Visual-Linguistic Model}
%The integration of a Visual-Linguistic Model (VLM) within the DCGAN framework represents a novel approach in our methodology. VLMs, which combine visual and textual data, enhance the generative capabilities of the network by providing additional context and semantic richness. This integration is expected to yield higher quality and more contextually relevant synthetic data.

% \TODO{Provide further details on implementation specifics, hyperparameter tuning, and evaluation metrics.}
