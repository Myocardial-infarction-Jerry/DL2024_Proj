\begin{abstract}
    Fine-Grained Visual Classification (FGVC) is a sophisticated computer vision task focused on distinguishing highly similar subcategories within broader categories. This challenge extends beyond general classification tasks, such as differentiating between cats and dogs, to identifying specific subcategories like bird species, dog breeds, or car models. Despite significant advancements, FGVC remains challenging due to the subtle differences between subcategories and the limited, often imbalanced, training data available.

    This paper addresses these challenges by integrating several advanced methodologies. We employ pretrained Vision Transformers (ViTs) for their superior ability to capture global dependencies and enhance initial model performance. To mitigate data limitations, we utilize Deep Convolutional Generative Adversarial Networks (DCGANs) for synthetic data generation, increasing training data diversity. Furthermore, we incorporate Visual-Linguistic Models (VLMs) within the DCGAN framework to enrich the contextual understanding of generated data.

    Our experiments, conducted on datasets such as CIFAR-10 and ImageNet, demonstrate that Vision Transformers outperform traditional convolutional neural networks like ResNet in both accuracy and loss convergence. The use of DCGANs and VLMs significantly boosts model performance by enhancing data diversity and semantic richness. Ablation studies confirm the critical importance of each component in our methodology.

    The proposed approach advances the state-of-the-art in FGVC by addressing key limitations and introducing innovative techniques. Future work will explore additional data augmentation strategies, hyperparameter sensitivity analysis, and the integration of more advanced models to further improve FGVC performance. Our contributions aim to push the boundaries of FGVC in both theoretical and practical applications.

    \TODO{Discuss the significance of FGVC in practical applications, review recent advancements in FGVC, highlight the unique challenges of FGVC compared to traditional image classification, and provide an overview of the proposed approach and its expected contributions to the field.}
\end{abstract}